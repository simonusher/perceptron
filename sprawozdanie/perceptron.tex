\documentclass{article}
\usepackage{graphicx}
\usepackage{amsmath}
\usepackage{hyperref}
\usepackage{float}
\usepackage{polski}
\usepackage[utf8]{inputenc}

\author{Szymon Woźniak, 235040}
\date{15.10.2019}
\title{Sprawozdanie 1\\Perceptron prosty i Adaline}


\begin{document}
	\pagenumbering{gobble}
	\maketitle
	\newpage
	\pagenumbering{arabic}
	
	\section{Opis badań eksperymentalnych}
	Celem badań jest poznanie właściwości dwóch modeli neuronu: perceptronu prostego i Adaline,
	oraz ich porównanie. Mają one również na celu przeanalizowanie wpływu doboru zakresu wag początkowych, współczynnika uczenia $\alpha$ oraz funkcji przejścia neuronu na szybkość uczenia.
	\section{Opis aplikacji wykorzystywanej do badań}
	Aplikacja do badań właściwości modeli neuronu została zaimplementowana w języku C++ w standardzie 17.
	\section{Charakterystyka zbiorów danych użytych do badań}
		Do przeprowadzenia badań zostały użyte zbiory danych
		skonstruowane z argumentów i wartości funkcji logicznych OR i AND.
		\begin{table}[H]
			\centering
			\caption{Funkcja logiczna OR}
			\label{tab:or-func}
			\begin{tabular}{|c|c|c|ll}
			\cline{1-3}
			$x$ & $y$ & $f(x, y)$ &  &  \\ \cline{1-3}
			0 & 0 & 0       &  &  \\ \cline{1-3}
			0 & 1 & 1       &  &  \\ \cline{1-3}
			1 & 0 & 1       &  &  \\ \cline{1-3}
			1 & 1 & 1       &  &  \\ \cline{1-3}
			\end{tabular}
			\end{table}
		\begin{table}[H]
			\centering
			\caption{Funkcja logiczna AND}
			\label{tab:and-func}
			\begin{tabular}{|c|c|c|ll}
			\cline{1-3}
			$x$ & $y$ & $f(x, y)$ &  &  \\ \cline{1-3}
			0 & 0 & 0       &  &  \\ \cline{1-3}
			0 & 1 & 1       &  &  \\ \cline{1-3}
			1 & 0 & 1       &  &  \\ \cline{1-3}
			1 & 1 & 1       &  &  \\ \cline{1-3}
			\end{tabular}
			\end{table}
	\section{Badania}

	\subsection*{Eksperyment 1. Wpływ zakresu początkowych wartości wag na szybkość uczenia}
	\paragraph{Założenia: } 
	\paragraph{Przebieg eksperymentu: }
	\paragraph{Wyniki: }
	\paragraph{Komentarz: }

	\subsection*{Eksperyment 2. Wpływ wartość współczynnika uczenia $\alpha$ na szybkość uczenia}
	\paragraph{Założenia: }
	\paragraph{Przebieg eksperymentu: }
	\paragraph{Wyniki: }
	\paragraph{Komentarz: }

	\subsection*{Eksperyment 3. Wpływ funkcji przejścia neuronu na szybkość uczenia}
	\paragraph{Założenia: }
	\paragraph{Przebieg eksperymentu: }
	\paragraph{Wyniki: }
	\paragraph{Komentarz: }
	\section{Podsumowanie}
\end{document}